\documentclass[12pt]{article} % use larger type; default would be 10pt
\usepackage[utf8]{inputenc} % set input encoding (not needed with XeLaTeX)
\usepackage{geometry} % to change the page dimensions
\geometry{letterpaper} % or letterpaper (US) or a5paper or....
% \geometry{margins=2in} % for example, change the margins to 2 inches all round
\usepackage{graphicx} % support the \includegraphics command and options
\usepackage{amsmath}
\usepackage[parfill]{parskip} % Activate to begin paragraphs with an empty line rather than an indent

%%% PACKAGES
\usepackage{booktabs} % for much better looking tables
\usepackage{array} % for better arrays (eg matrices) in maths
\usepackage{paralist} % very flexible & customisable lists (eg. enumerate/itemize, etc.)
\usepackage{verbatim} % adds environment for commenting out blocks of text & for better verbatim
\usepackage{subfig} % make it possible to include more than one captioned figure/table in a single float
% These packages are all incorporated in the memoir class to one degree or another...

%%% HEADERS & FOOTERS
\usepackage{fancyhdr} % This should be set AFTER setting up the page geometry
\pagestyle{fancy} % options: empty , plain , fancy
\renewcommand{\headrulewidth}{0pt} % customise the layout...
\lhead{}\chead{}\rhead{}
\lfoot{}\cfoot{\thepage}\rfoot{}

%%% SECTION TITLE APPEARANCE
\usepackage{sectsty}
\allsectionsfont{\sffamily\mdseries\upshape} % (See the fntguide.pdf for font help)
% (This matches ConTeXt defaults)

%%% ToC (table of contents) APPEARANCE
\usepackage[nottoc,notlof,notlot]{tocbibind} % Put the bibliography in the ToC
\usepackage[titles,subfigure]{tocloft} % Alter the style of the Table of Contents
\renewcommand{\cftsecfont}{\rmfamily\mdseries\upshape}
\renewcommand{\cftsecpagefont}{\rmfamily\mdseries\upshape} % No bold!

%%% END Article customizations

%%% The "real" document content comes below...

\title{Brief Article}
\author{The Author}
%\date{} % Activate to display a given date or no date (if empty),
         % otherwise the current date is printed

\begin{document}

Given $\Delta^n y$ is the $n^{th}$ forward difference of sequence $y$
\begin{equation}
\Delta f(x)=f(x+1)-f(x)
\end{equation}

And $x^{\underline n}$ is the falling sequential product (falling factorial, factorial power)
\begin{equation}
x^{\underline n}=\overbrace{x(x-1)\ldots(x-n+1)}^{n~\mathrm{factors}}\qquad\mbox{for integer}~n\ge0
\end{equation}

And $C_n$ is the quotient of the $0^{th}$ element of $\Delta^n y$ over $n!$
\begin{equation}
C_n = \frac{\Delta^n y[0]}{n!}
\end{equation}

Where $y = \langle 0,1,4,9,16,25,36,49,64,81,\ldots\rangle$\\
\begin{align}
\notag \Delta^0 y &= \langle 0,1,4,9,16,25,36,49,64,81,\ldots\rangle\\
\Delta^1 y &= \langle 1,3,5,7,9,11,13,15,17,\ldots\rangle\\
\notag \Delta^2 y &= \langle 2,2,2,2,2,2,2,2,\ldots\rangle\\
\notag \Delta^3 y &= \langle 0,0,0,0,0,0,0,\ldots\rangle
\end{align}

\begin{align}
\notag C &= \langle\frac{0}{0!},\frac{1}{1!},\frac{2}{2!}\rangle\\
&= \langle\frac{0}{1},\frac{1}{1},\frac{2}{2}\rangle\\
\notag &= \langle0,1,1\rangle
\end{align}

\begin{align}
\notag x^{\underline 0} &= 0\\
x^{\underline 1} &= x\\
\notag x^{\underline 2} &= x(x-1) = x^2 - x
\end{align}

\begin{align}
G &= {\displaystyle\sum C_n x^{\underline{n}}}\\
\notag &= (0 \cdot 0) + (1 \cdot x) + (1 \cdot (x^2 - x))\\
\notag &= x + x^2 - x\\
\notag &= x^2
\end{align}

\end{document}
